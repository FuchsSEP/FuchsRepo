\documentclass[a4paper]{scrreprt}
\usepackage[german]{babel}
\usepackage[german]{translator}
\usepackage[utf8]{inputenc}
\usepackage[T1]{fontenc}
\usepackage{blindtext} 




\usepackage{ae}
\usepackage[bookmarks,bookmarksnumbered]{hyperref}
\usepackage{graphicx}
\usepackage{color}
\usepackage[dvipsnames]{xcolor}
\begin{document}
	
	\newcommand{\class}[1]{\paragraph{Klasse #1:}\ \\ }
	\newcommand{\method}[1]{\textcolor{blue}{#1}}
	\newcommand{\kursiv}[1]{{\it #1}}
	\newcommand{\override}{{\it @Override}\ \\}
	
	\chapter{Klassenbeschreibung und Klassendiagramm}
	In diesem Kapitel werden alle Klassen der Anwendung \textbf{ofCourse} aufgeführt.
	Um eine bessere Übersicht und Strukturierung zu erhalten wird das ganze Projekt in Pakete aufgeteilt.
	
	\section{Package Action}
	\subsection{Klasse SessionUser}
	\kursiv{ManagedBean, SessionScoped}\\
	Die Klasse speichert Informationen über die Session eines Benutzers. Gespeichert werden die ID des Benutzers, dessen Benutzerrolle, die gewählte Sprache.
	\begin{itemize}
		\item \method{public int getUserID()}
		\item \method{public void setUserId(int userID)}
		\item \method{public String getUserRole()}
		\item \method{public void setUserRole(String userRole)}
		\item \method{public String getLanguage()}
		\item \method{public void setLanguage(String language)}
	\end{itemize}
	
	\section{Package Model}
	
	\section{Package Services}
	
	\section{Package Database}
	\subsection{Klasse DatabaseConnectionManager}
	Die Klasse ist zuständig für das Aufbauen der Verbindungen zur Datenbank. Des Weiteren
	speichert und verwaltet sie die Datenbankverbindungen. 
	\begin{itemize}
		\item \method{public synchronized Connection getConnection()}
		\item \method{public synchronized void releaseConnection()}
		\item \method{public static DatabaseConnectionManager getInstance()}
		\item \method{public void shutDown()}
	\end{itemize}
	
	\section{Package System}
	
		\subsection{Klasse CheckPhase}
		Die Klasse ist zuständig für die Überprüfung, ob der jeweilige Benutzer die Berechtigung besitzt, auf die angeforderte Seite zu gelangen.\\
		Die Klasse implementiert das Interface PhaseListener.
		\begin{itemize}
			\item \method{public PhaseId getPhaseId()}
			\item \override
			\method{public void beforePhase(PhaseEvent arg0)}
			\item \override
			\method{public void afterPhase(PhaseEvent event)}	
		\end{itemize}
		
		\subsection{Klasse Maintenance}
		Die Klasse ist zuständig dafür, dass Kurse sechs Monate nach ihrem Enddatum automatisch aus dem System gelöscht werden.\\
		Die Klasse implementiert das Interface Runnable.
		\begin{itemize}
			\item \method{public boolean isMaintenaceStopped()}
		 	\item \method{public synchronized void shutDown()}
		 	\item \method{public static Maintenance getInstance()}
		 	\item \override
		 	\method{public void run()}
		\end{itemize}
		
		\subsection{LaunchSystem}
		\kursiv{ManagedBean, ApplicationScoped}\\
		Die Klasse ist zuständig für das Starten des System. Sie startet den Maintenance - Thread, lädt Properties und stellt die Datenbankverbindung her.
		\begin{itemize}
			\item \kursiv{@PostConstruct}\\
			\method{public void startSystem()}
			\item \kursiv{@PreDestroy}\\
			\method{public void shutdownMaintenance()}
		\end{itemize}
		
		
		
	\section{Package Util}
	
	\subsection{Klasse Mail}
	\kursiv{ManagedBean, ApplicationScoped}\\
	Die Klasse ist für die E-Mailbenachrichtigung der Benutzer zuständig. Sie ist sowohl für die automatisch gesendeten E-Mails, wie unter anderem die Verifizierung oder Accountaktivierungsbestätigung, als auch für die von Kursleitern gesendeten Mails zuständig.
	\begin{itemize}
		\item \method{public boolean sendVerificationMail(int UserID)}
		\item \method{public boolean sendConfirmaitionMail(int UserID)}
		\item \method{public boolean sendMail(String subject, String message, ArrayList<String> mailRecipients)}
	\end{itemize}
	
	\subsection{Klasse PasswordHash}
	Diese Klasse ist zuständig für das Hashen der Passwörter.
	\begin{itemize}
		\item \method{public static String hash(String password, int salt)}
	\end{itemize}
	
	\subsection{Klasse LanguageMangager}
	Diese Klasse ist zuständig für die angezeigte Systemsprache. In ihr werden die unterstützten Sprachen verwaltet und sie ist zuständig für das Auslesen der Anzeigetexte aus der .properties - Datei der gewählten Sprache.
	\begin{itemize}
		\item \method{public static LanguageManager getInstance()}
		\item \method{public LinkedHashMap<String, Object> getSupportedLanguages()}
		\item \method{public String getProperty(String key)}
		\item \method{public void switchLanguage(String language)}
	\end{itemize}
	
	\subsection{Klasse PropertyManager}
	Die Klasse ist zuständig für das Auslesen und Bearbeiten der Property - Datei, welche die Daten für die Systemkonfiguration, also die Daten für die Datenbankverbindung und den EMail - Service enthält.
	\begin{itemize}
		\item \method{public static PropertyManager getInstance()}
		\item \method{public String getProperty(String key)}
		\item \method{public String setProperty(String key)}
	\end{itemize}
	
	\subsection{Klasse IDGenerator}
	Die Klasse ist zuständig für das Generieren einer im System eindeutigen Identifikationsnummer für einen Benutzer, einen Kurs oder eine Kurseinheit.
	\begin{itemize}
		\item \method{public static int generateUserID()}
		\item \method{public static int generateCourseID()}
		\item \method{public static int generateCourseUnitID(int CourseID)}
	\end{itemize}
	
	\subsection{Klasse RandomSaltGenerator}
	Die Klasse ist zuständig für das Generieren eines zufälligen Salt, welcher für das Hashen des Benutzerpassworts benötigt wird.
	\begin{itemize}
		\item \method{public static int generateRandomSalt()}
	\end{itemize} 

	\section{Package Exception}
	\subsection{BankAccountException}
	Sind Fehler, welche beim Aufladen des Guthabenkontos auftreten. 
	
	\section{Package Validator}
	Dieser Abschnitt beschäftigt sich mit den benötigten Validatoren, die notwendig sind, um die Eingaben des Benutzers zu überprüfen und gegebenenfalls Fehlermeldungen zu generieren.\\
	Alles Klassen dieses Packages implementieren das Interface Validator.
		
	\subsection{Klasse UserNameValidator}
	Der Validator überprüft, ob der eingegebene Benutzername schon im System vergeben ist.
	\begin{itemize}
		\item \override
		\method{public void validate()}
	\end{itemize}
	
	\subsection{Klasse EMailValidator}
	Der Validator überprüft, ob die Eingabe ein gültiges E-Mail-Format besitzt und ob die eingegebene E-Mailadresse bereits im System existiert.
	\begin{itemize}
		\item \override
		\method{public void validate()}
	\end{itemize}
	
	\subsection{Klasse PasswordValidator}
	Dieser Validator überprüft, ob das eingegebene Passwort gewisse Sicherheitsanforderungen bezüglich Länge und Zeichenwahl erfüllt. Vorgesehen Anforderungen an das Passwort sind mindestens 8 Zeichen, mindestens ein Sonderzeichen, mindestens eine Ziffer, Verwendung von Groß- und Kleinbuchstaben. Außerdem dürfen im Passwort keine Umlaute sowie kein 'ß' vorkommen.
	\begin{itemize}
		\item \override
		\method{public void validate()}
	\end{itemize}
	
	\subsection{Klasse ConfirmPasswordValidator}
	Dieser Validator überprüft zwei Passwörter auf ihre Übereinstimmung.
	\begin{itemize}
		\item \override
		\method{public void validate()}
	\end{itemize}
	
	\subsection{Klasse BirthdateValidator}
	Der Validator überprüft, ob das eingegebene Datum eingegebene Datum in der Zukunft liegt und ob es mehr als 150 Jahre zurückliegt.
	\begin{itemize}
		\item \override
		\method{public void validate()}
	\end{itemize}
	
	\subsection{Klasse UserImageValidator}
	Dieser Validator überprüft eine Bilddatei auf die richtige Dateiendung .jpg. Zusätzlich wird überprüft, ob
	die maximale Dateigröße und die maximal zugelassene Auflösung für ein Profilbild eines Benutzers eingehalten wird.
	\begin{itemize}
		\item \override
		\method{public void validate()}
	\end{itemize}
	
    \subsection{Klasse DateValidator}
    Der Validator überprüft Datumseingaben.
    \begin{itemize}
	    \item \override
	    \method{public void validate()}
    \end{itemize}
    
    \subsection{Klasse CreditCardValidator}
    Der Validator überprüft, ob eine Kreditkartenummer gültig ist.
    \begin{itemize}
    	\item \override
    	\method{public void validate()}
    \end{itemize}
    
    \subsection{Klasse CVCValidator}
    Dieser Validator überprüft eine CVC Nummer auf ihre Gültigkeit.
    \begin{itemize}
        \item \override
     	\method{public void validate()}
    \end{itemize}
    
    \subsection{Klasse OfflineTransactionValidator}
    Dieser Validator überprüft, ob bei einer Offline-Aufladung des Guthabenkontos eines Benutzers der eingegebene Name und die eingegebenen Benutzeridentifikationsnummer auch zum selben Benutzer gehören.
    \begin{itemize}
    	\item \override
    	\method{public void validate()}
    \end{itemize}
    
    \subsection{Klasse PriceValidator}
    Der Validator überprüft, ob der eingegebene Preis das korrekte Format hat, dass heißt, ob die Zahl nicht-negativ ist und zwei Nachkommastellen besitzt.
    \begin{itemize}
     	\item \override
     	\method{public void validate()}
    \end{itemize}
	
	\subsection{Klasse LogoImageValidator}
	Dieser Validator überprüft eine Bilddatei auf die richtige Dateiendung .jpg. Zusätzlich wird überprüft, ob
	die maximale Dateigröße und die maximal zugelassene Auflösung für ein Anwendungslogo eingehalten wird.
	\begin{itemize}
		\item \override
		\method{public void validate()}
	\end{itemize}
	
	\subsection{Klasse CustomStyleCSSValidator}
	Der Validator überprüft, ob die Datei den Namen 'customStyle' besitzt und ob es sich um den Dateityp mit der Endung .css handelt.
	\begin{itemize}
		\item \override
		\method{public void validate()}
	\end{itemize}
	
    

	
	\section{Verwendete Libraries}
	In dem System verwendete Libraries:
	\begin{itemize}
		\item Commons Fileupload: Library für das Hochladen von Dateien.
		\item JavaMail: Library für das Versenden von E-Mails.
		\item JFreeChart: Library für die Erstellung von Diagrammen.
		\item Log4J: Library für das Loggen von Meldungen.
	\end{itemize}
	\section{Klassendiagramm}
    
	
	
	
	
\end{document}